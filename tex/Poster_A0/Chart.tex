\linespread{1.8}\selectfont
\begin{minipage}{245mm}
\begin{center} \end{center}
\begin{tabular}{l}\begin{screen}\linespread{1.2}\selectfont
1. Read data of  sample-nodes.
\end{screen}\end{tabular}
\begin{center}\vspace{-5mm}$|$\vspace{-3mm}\end{center}
\begin{tabular}{l}\begin{screen}\linespread{1.2}\selectfont
2. \hspace{-2mm}Place cluster-nodes at random \hspace{1mm}on \hspace{1mm}the same space as sample-nodes.
\end{screen}\end{tabular}
\begin{center}\vspace{-5mm}$|$\vspace{-3mm}\end{center}
\begin{tabular}{l}\begin{screen}\linespread{1.2}\selectfont
3. Assign each sample-node to the nearest cluster-node.
\end{screen}\end{tabular}
\put(-5,5){\line(1,0){10}\line(0,-1){428}}\vspace{-402pt}
\begin{center}\vspace{-5mm}$|$\vspace{-3mm}\end{center}
\begin{tabular}{l}\begin{screen}\linespread{1.2}\selectfont
4. \hspace{-5mm}Calculate each coordinate for a centroid of a set of sample-nodes which are assigned to the particular cluster-node.
\end{screen}\end{tabular}
\begin{center}\vspace{-5mm}$|$\vspace{-3mm}\end{center}
\begin{tabular}{l}\begin{screen}\linespread{1.2}\selectfont
5. Move each cluster-node toward the given centroid.
\end{screen}\end{tabular}
\begin{center}\vspace{-5mm}$|$\vspace{-3mm}\end{center}
\begin{tabular}{l}\begin{screen}\linespread{1.2}\selectfont
6. \hspace{-6mm}Calculate \hspace{2mm}the \hspace{2mm}distance between \hspace{1mm}given \hspace{1mm}two cluster-nodes.
\end{screen}\end{tabular}
\begin{center}\vspace{-5mm}$|$\vspace{-3mm}\end{center}
\begin{tabular}{l}\begin{screen}\linespread{1.2}\selectfont
7. Unify a pair of cluster-nodes with the shortest distance, Unless the distance is longer than a threshold value given by the user.
\end{screen}\end{tabular}
\put(-5,5){\line(1,0){10}}
\begin{center}\vspace{-3mm}\end{center}
\vspace{-5mm}
\begin{tabular}{l}\linespread{1.2}\selectfont
\ Repeat steps 3 to 7.
\end{tabular}
\end{minipage}
